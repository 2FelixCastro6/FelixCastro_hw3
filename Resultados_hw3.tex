
\documentclass[12pt]{article}
\usepackage{graphicx}
\usepackage{float}

\begin{document}


\textbf{\centering Resultados de la solucion a la ecuacion de onda y el Sistema solar }

\textbf{Félix Roberto Castro López\ \ 201518139}


\section{Ecuacion de onda}

 A partir de la solucion de la ecuacion de onda, realizamos una grafica en la cual se observa la propagacion de la onda para el instante de tiempo medio y final.


\begin{figure}[H]
\centering\includegraphics[width=0.8\linewidth]{t30.pdf}
\caption{Propagacion de la onda para \textbf{T:30}}
\end{figure}

\begin{figure}[H]
\centering\includegraphics[width=0.8\linewidth]{t60.pdf}
\caption{Propagacion de la onda para \textbf{T:60}}
\end{figure}




\section{Sistema solar}
\begin{equation}
\label{eq:emc}
F_{i}=G \sum_{i\not\equiv j}^{N}\frac{m_{i}m_{j}}{r_{ij}}(\vec{r_{j}}-\vec{r_{j}}) 
\end{equation}

En el sistema solar la interaccion gravitacional entre los diferentes planetas se puede describir con la ecuacion 1 la cual nos dice una relacion cuantitativa de la fuerza entre dos cuerpos con masa. De esta manera, a partir de la aceleracion obtenida con esta ecuacion se halla la posicion y posteriormente se grafica para obtener las orbitas de cada planeta.  

\begin{figure}[H]
%orbitas pdf seria si no tuviera errores el script .c
%\centering\includegraphics[width=0.8\linewidth]{orbitas.pdf}
\centering\includegraphics[width=0.8\linewidth]{t30.pdf}
\caption{Orbitas del sistema solar}
\end{figure}

\end{document}
